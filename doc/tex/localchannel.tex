\subsection{Introduction}

chan\_local is a pseudo-channel. Use of this channel simply loops calls back
into the dialplan in a different context. Useful for recursive routing.

\subsection{Syntax}
\begin{verbatim}
 Local/extension@context[/{n|j}]
\end{verbatim}

Adding "/n" at the end of the string will make the Local channel not do a
native transfer (the "n" stands for "n"o release) upon the remote end answering
the line. This is an esoteric, but important feature if you expect the Local
channel to handle calls exactly like a normal channel. If you do not have the
"no release" feature set, then as soon as the destination (inside of the Local
channel) answers the line and one audio frame passes, the variables and dial plan
will revert back to that of the original call, and the Local channel will become a
zombie and be removed from the active channels list. This is desirable in some
circumstances, but can result in unexpected dialplan behavior if you are doing
fancy things with variables in your call handling.

There is another option that can be used with local channels, which is the "j"
option.  The "j" option must be used with the "n" option to make sure that the
local channel does not get optimized out of the call.  This option will enable
a jitterbuffer on the local channel.  The jitterbuffer will be used to de-jitter
audio that it receives from the channel that called the local channel.  This is
especially in the case of putting chan\_local in between an incoming SIP call
and Asterisk applications, so that the incoming audio will be de-jittered.

Using the "m" option will cause chan_local to forward music on hold start and stop
requests. Normally chan_local acts on them and it is started or stopped on the
Local channel itself.

\subsection{Purpose}

The Local channel construct can be used to establish dialing into any part of
the dialplan.

Imagine you have a TE410P in your box. You want to do something for which you
must use a Dial statement (for instance when dropping files in
\path{/var/spool/outgoing}) but you do want to be able to use your dialplans
least-cost-routes or other intelligent stuff. What you could do before we had
chan\_local was create a cross-link between two ports of the TE410P and then
Dial out one port and in the other. This way you could control where the call
was going.

Of course, this was a nasty hack, and to make it more sensible, chan\_local was
built.

The "Local" channel driver allows you to convert an arbitrary extension into a
channel. It is used in a variety of places, including agents, etc.

This also allows us to hop to contexts like a GoSub routine; See examples below.

\subsection{Examples}
\begin{astlisting}
\begin{verbatim}
[inbound] ; here falls all incoming calls
exten => s,1,Answer
exten => s,2,Dial(local/200@internal,30,r)
exten => s,3,Playback(sorrynoanswer)
exten => s,4,Hangup

[internal] ; here where our phones falls for default
exten => 200,1,Dial(sip/blah)
exten => 200,102,VoiceMail(${EXTEN}@default)

exten => 201,1,Dial(DAHDI/1)
exten => 201,102,VoiceMail(${EXTEN}@default)

exten => _0.,1,Dial(DAHDI/g1/${EXTEN:1}) ; outgoing calls with 0+number
\end{verbatim}
\end{astlisting}

\subsection{Caveats}

If you use chan\_local from a call-file and you want to pass channel variables
into your context, make sure you append the '/n', because otherwise
chan\_local will 'optimize' itself out of the call-path, and the variables will
get lost. i.e.

\begin{verbatim}
 Local/00531234567@pbx becomes Local/00531234567@pbx/n
\end{verbatim}
