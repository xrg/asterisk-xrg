\subsubsection{Introduction}

Asterisk support different QoS settings on application level on various protocol
on any of signaling and media. Type of Service (TOS) byte can be set on
outgoing IP packets for various protocols. The TOS byte is used by the network
to provide some level of Quality of Service (QoS) even if the network is
congested with other traffic.

Also asterisk running on Linux can set 802.1p CoS marks in VLAN packets for all
used VoIP protocols. It is useful when you are working in switched environment.
In fact asterisk only set priority for Linux socket. For mapping this priority
and VLAN CoS mark you need to use this command:

\begin{verbatim}
vconfig set_egress_map [vlan-device] [skb-priority] [vlan-qos]
\end{verbatim}

In table behind shown all voice channels and other modules of asterisk, that
support QoS settings for network traffic and type of traffic which can have
QoS settings.

\begin{verbatim}
 Channel Drivers
+==============+===========+=====+=====+=====+
|              | Signaling |Audio|Video| Text|
+==============+===========+=====+=====+=====+
|chan_sip      |     +     |  +  |  +  |  +  |
|--------------+-----------+-----+-----+-----+
|chan_skinny   |     +     |  +  |  +  |     |
|--------------+-----------+-----+-----+-----+
|chan_mgcp     |     +     |  +  |     |     |
|--------------+-----------+-----+-----+-----+
|chan_unistim  |     +     |  +  |     |     |
|--------------+-----------+-----+-----+-----+
|chan_h323     |           |  +  |     |     |
|--------------+-----------+-----+-----+-----+
|chan_iax2     |            +                |
+==============+=============================+
 Other
+==============+=============================+
| dundi.conf   |     + (tos setting)         |
|--------------+-----------------------------+
| iaxprov.conf |     + (tos setting)         |
+==============+=============================+
\end{verbatim}


\subsubsection{IP TOS values}

The allowable values for any of the tos* parameters are: 
CS0, CS1, CS2, CS3, CS4, CS5, CS6, CS7, AF11, AF12, AF13, AF21, AF22, AF23, 
AF31, AF32, AF33, AF41, AF42, AF43 and ef (expedited forwarding),

The tos* parameters also take numeric values.

Note, that on Linux system you can use ef value in case your asterisk is running
from a user other then root only when you have compiled asterisk with libcap.

The lowdelay, throughput, reliability, mincost, and none values are removed
in current releases.

\subsubsection{802.1p CoS values}

As far as 802.1p uses 3 bites from VLAN header, there are parameter can take
integer values from 0 to 7.

\subsubsection{Recommended values}
Recommended values shown above and also included in sample configuration files:
\begin{verbatim}
+============+=========+======+
|            |  tos    |  cos |
+============+=========+======+
|Signaling   |  cs3    |  3   |
|Audio       |  ef     |  5   |
|Video       |  af41   |  4   |
|Text        |  af41   |  3   |
|Other       |  ef     |      |
+============+=========+======+
\end{verbatim}

\subsubsection{IAX2}

In iax.conf, there is a "tos" parameter that sets the global default TOS
for IAX packets generated by chan\_iax2.  Since IAX connections combine
signalling, audio, and video into one UDP stream, it is not possible
to set the TOS separately for the different types of traffic.

In iaxprov.conf, there is a "tos" parameter that tells the IAXy what TOS
to set on packets it generates.  As with the parameter in iax.conf,
IAX packets generated by an IAXy cannot have different TOS settings
based upon the type of packet.  However different IAXy devices can
have different TOS settings.

\subsubsection{SIP}

In sip.conf, there are three parameters that control the TOS settings:
"tos\_sip", "tos\_audio", "tos\_video" and "tos\_text". tos\_sip controls
what TOS SIP call signaling packets are set to. tos\_audio, tos\_video
and tos\_text controls what TOS RTP audio, video or text accordingly
packets are set to.

There are four parameters to control 802.1p CoS: "cos\_sip", "cos\_audio",
"cos\_video" and "cos\_text". It behavior the same as written above.

\subsubsection{Other RTP channels}

chan\_mgcp, chan\_h323, chan\_skinny and chan\_unistim also support TOS and
CoS via setting tos and cos parameters in correspond to module config 
files. Naming style and behavior same as for chan\_sip.

\subsubsection{Reference}

IEEE 802.1Q Standard:
\url{http://standards.ieee.org/getieee802/download/802.1Q-1998.pdf}
Related protocols: IEEE 802.3, 802.2, 802.1D, 802.1Q

RFC 2474 - "Definition of the Differentiated Services Field
(DS field) in the IPv4 and IPv6 Headers", Nichols, K., et al,
December 1998.

IANA Assignments, DSCP registry
Differentiated Services Field Codepoints
\url{http://www.iana.org/assignments/dscp-registry}

To get the most out of setting the TOS on packets generated by
Asterisk, you will need to ensure that your network handles packets
with a TOS properly.  For Cisco devices, see the previously mentioned
"Enterprise QoS Solution Reference Network Design Guide".  For Linux
systems see the "Linux Advanced Routing \& Traffic Control HOWTO" at 
\url{http://www.lartc.org/}.

For more information on Quality of
Service for VoIP networks see the "Enterprise QoS Solution Reference
Network Design Guide" version 3.3 from Cisco at:
\url{http://www.cisco.com/application/pdf/en/us/guest/netsol/ns432/c649/ccmigration\_09186a008049b062.pdf}
