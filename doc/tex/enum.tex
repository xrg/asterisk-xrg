\section{The ENUMLOOKUP dialplan function}

The ENUMLOOKUP function is more complex than it first may appear, and
this guide is to give a general overview and set of examples that may
be well-suited for the advanced user to evaluate in their
consideration of ENUM or ENUM-like lookup strategies.  This document
assumes a familiarity with ENUM (RFC3761) or ENUM-like methods, as
well as familiarity with NAPTR DNS records (RFC2915, RFC3401-3404).
For an overview of NAPTR records, and the use of NAPTRs in the ENUM
global phone-number-to-DNS mapping scheme, please see
\url{http://www.voip-info.org/tiki-index.php?page=ENUM} for more detail.

Using ENUM within Asterisk can be simple or complex, depending on how
many failover methods and redundancy procedures you wish to utilize.
Implementation of ENUM paths is supposedly defined by the person
creating the NAPTR records, but the local administrator may choose to
ignore certain NAPTR response methods (URI types) or prefer some over
others, which is in contradiction to the RFC.  The ENUMLOOKUP method
simply provides administrators a method for determining NAPTR results
in either the globally unique ENUM (e164.arpa) DNS tree, or in other
ENUM-like DNS trees which are not globally unique.  The methods to
actually create channels ("dial") results given by the ENUMLOOKUP
function is then up to the administrator to implement in a way that
best suits their environment.

\begin{verbatim}
Function: ENUMLOOKUP(number[,Method-type[,options[,record#[,zone-suffix]]]])
\end{verbatim}

  Performs an ENUM tree lookup on the specified number, method type, and
  ordinal record offset, and returns one of four different values:

\begin{enumerate}
   \item post-parsed NAPTR of one method (URI) type
   \item count of elements of one method (URI) type
   \item count of all method types
   \item full URI of method at a particular point in the list of all possible methods
\end{enumerate}

\subsection{Arguments}

\begin{itemize}
  \item number
  \begin{itemize}
    \item telephone number or search string.  Only numeric values
    within this string are parsed; all other digits are ignored for
    search, but are re-written during NAPTR regexp expansion.
  \end{itemize}

  \item service\_type
  \begin{itemize}
     \item tel, sip, h323, iax2, mailto, ...[any other string],
     ALL. Default type is "sip".
     Special name of "ALL" will create a list of method types across
     all NAPTR records for the search number, and then put the results
     in an ordinal list starting with 1. The position <number>
     specified will then be returned, starting with 1 as the first
     record (lowest value) in the list.  The service types are not
     hardcoded in Asterisk except for the default (sip) if no other
     service type specified; any method type string (IANA-approved or
     not) may be used except for the string "ALL".
  \end{itemize}

  \item options
  \begin{itemize}
    \item c
    \begin{itemize}
      \item count. Returns the number of records of this type are returned
    (regardless of order or priority.)  If "ALL" is the specified
    service\_type, then a count of all methods will be returned for the
    DNS record.
    \end{itemize}
  \end{itemize}

  \item record\#
  \begin{itemize}
    \item which record to present if multiple answers are returned
    <integer> = The record in priority/order sequence based on the
    total count of records passed back by the query. If a service\_type
    is specified, all entries of that type will be sorted into an
    ordinal list starting with 1 (by order first, then priority).
    The default of <options> is "1"
  \end{itemize}

  \item zone\_suffix
  \begin{itemize}
    \item allows customization of the ENUM zone. Default is e164.arpa.
  \end{itemize}
\end{itemize}

\subsection{Examples}

Let's use this ENUM list as an example (note that these examples exist
in the DNS, and will hopefully remain in place as example
destinations, but they may change or become invalid over time.  The
end result URIs are not guaranteed to actually work, since some of
these hostnames or SIP proxies are imaginary.  Of course, the tel:
replies go to directory assistance for New York City and San
Francisco...)  Also note that the complex SIP NAPTR at weight 30 will
strip off the leading "+" from the dialed string if it exists.  This
is probably a better NAPTR than hard-coding the number into the NAPTR,
and it is included as a more complex regexp example, though other
simpler NAPTRs will work just as well.

\begin{verbatim}
0.2.0.1.1.6.5.1.0.3.1.loligo.com. 3600 IN NAPTR 10 100 "u"
     "E2U+tel" "!^\\+13015611020$!tel:+12125551212!" .
0.2.0.1.1.6.5.1.0.3.1.loligo.com. 3600 IN NAPTR 21 100 "u"
     "E2U+tel" "!^\\+13015611020$!tel:+14155551212!" .
0.2.0.1.1.6.5.1.0.3.1.loligo.com. 3600 IN NAPTR 25 100 "u"
     "E2U+sip" "!^\\+13015611020$!sip:2203@sip.fox-den.com!" .
0.2.0.1.1.6.5.1.0.3.1.loligo.com. 3600 IN NAPTR 26 100 "u"
     "E2U+sip" "!^\\+13015611020$!sip:1234@sip-2.fox-den.com!" .
0.2.0.1.1.6.5.1.0.3.1.loligo.com. 3600 IN NAPTR 30 100 "u"
     "E2U+sip" "!^\\+*([^\\*]*)!sip:\\1@sip-3.fox-den.com!" .
0.2.0.1.1.6.5.1.0.3.1.loligo.com. 3600 IN NAPTR 55 100 "u"
     "E2U+mailto" "!^\\+13015611020$!mailto:jtodd@fox-den.com!" .
\end{verbatim}

Example 1: Simplest case, using first SIP return (use all defaults
except for domain name)
\begin{verbatim}
exten => 100,1,Set(foo=${ENUMLOOKUP(+13015611020,,,,loligo.com)})
  returns: ${foo}="2203@sip.fox-den.com"
\end{verbatim}

Example 2: What is the first "tel" pointer type for this number?
(after sorting by order/preference; default of "1" is assumed in
options field)
\begin{verbatim}
exten => 100,1,Set(foo=${ENUMLOOKUP(+13015611020,tel,,,loligo.com)})
  returns: ${foo}="+12125551212"
\end{verbatim}

Example 3: How many "sip" pointer type entries are there for this number?
\begin{verbatim}
exten => 100,1,Set(foo=${ENUMLOOKUP(+13015611020,sip,c,,loligo.com)})
  returns: ${foo}=3
\end{verbatim}

Example 4: For all the "tel" pointer type entries, what is the second
one in the list? (after sorting by preference)
\begin{verbatim}
exten => 100,1,Set(foo=${ENUMLOOKUP(+13015611020,tel,,2,loligo.com)})
  returns: ${foo}="+14155551212"
\end{verbatim}

Example 5: How many NAPTRs (tel, sip, mailto, etc.) are in the list for this number?
\begin{verbatim}
exten => 100,1,Set(foo=${ENUMLOOKUP(+13015611020,ALL,c,,loligo.com)})
  returns: ${foo}=6
\end{verbatim}

Example 6: Give back the second full URI in the sorted list of all NAPTR URIs:
\begin{verbatim}
exten => 100,1,Set(foo=${ENUMLOOKUP(+13015611020,ALL,,2,loligo.com)})
  returns: ${foo}="tel:+14155551212"  [note the "tel:" prefix in the string]
\end{verbatim}

Example 7: Look up first SIP entry for the number in the e164.arpa zone (all defaults)
\begin{verbatim}
exten => 100,1,Set(foo=${ENUMLOOKUP(+437203001721)})
  returns: ${foo}="enum-test@sip.nemox.net"  [note: this result is
  subject to change as it is "live" DNS and not under my control]
\end{verbatim}

Example 8: Look up the ISN mapping in freenum.org alpha test zone
\begin{verbatim}
exten => 100,1,Set(foo=${ENUMLOOKUP(1234*256,,,,freenum.org)})
  returns: ${foo}="1234@204.91.156.10"  [note: this result is subject
  to change as it is "live" DNS]
\end{verbatim}

Example 9: Give back the first SIP pointer for a number in the
\begin{verbatim}
enum.yoydynelabs.com zone (invalid lookup)
exten => 100,1,Set(foo=${ENUMLOOKUP(1234567890,sip,,1,enum.yoyodynelabs.com)})
  returns: ${foo}=""
\end{verbatim}

\subsection{Usage notes and subtle features}
\begin{itemize}
  \item The use of "+" in lookups is confusing, and warrants further
  explanation.  All E.164 numbers ("global phone numbers") by
  definition need a leading "+" during ENUM lookup.  If you neglect to
  add a leading "+", you may discover that numbers that seem to exist
  in the DNS aren't getting matched by the system or are returned with
  a null string result.  This is due to the NAPTR reply requiring a
  "+" in the regular expression matching sequence.  Older versions of
  Asterisk add a "+" from within the code, which may confuse
  administrators converting to the new function.  Please ensure that
  all ENUM (e164.arpa) lookups contain a leading "+" before lookup, so
  ensure your lookup includes the leading plus sign.  Other DNS trees
  may or may not require a leading "+" - check before using those
  trees, as it is possible the parsed NAPTRs will not provide correct
  results unless you have the correct dialed string.  If you get
  console messages like "WARNING[24907]: enum.c:222 parse\_naptr: NAPTR
  Regex match failed." then it is very possible that the returned
  NAPTR expects a leading "+" in the search string (or the returned
  NAPTR is mis-formed.)

  \item If a query is performed of type "c" ("count") and let's say you
  get back 5 records and then some seconds later a query is made
  against record 5 in the list, it may not be the case that the DNS
  resolver has the same answers as it did a second or two ago - maybe
  there are only 4 records in the list in the newest query.  The
  resolver should be the canonical storage location for DNS records,
  since that is the intent of ENUM.  However, some obscure future
  cases may have wildly changing NAPTR records within several seconds.
  This is a corner case, and probably only worth noting as a very rare
  circumstance. (note: I do not object to Asterisk's dnsmgr method of
  locally caching DNS replies, but this method needs to honor the TTL
  given by the remote zone master.  Currently, the ENUMLOOKUP function
  does not use the dnsmgr method of caching local DNS replies.)

  \item If you want strict NAPTR value ordering, then it will be
  necessary to use the "ALL" method to incrementally step through the
  different returned NAPTR pointers.  You will need to use string
  manipulation to strip off the returned method types, since the
  results will look like "sip:12125551212" in the returned value.
  This is a non-trivial task, though it is required in order to have
  strict RFC compliance and to comply with the desires of the remote
  party who is presenting NAPTRs in a particular order for a reason.

  \item Default behavior for the function (even in event of an error) is
  to move to the next priority, and the result is a null value.  Most
  ENUM lookups are going to be failures, and it is the responsibility
  of the dialplan administrator to manage error conditions within
  their dialplan.  This is a change from the old app\_enumlookup method
  and it's arbitrary priority jumping based on result type or failure.

  \item Anything other than digits will be ignored in lookup strings.
  Example: a search string of "+4372030blah01721" will turn into
  1.2.7.1.0.0.3.0.2.7.3.4.e164.arpa. for the lookup.  The NAPTR
  parsing may cause unexpected results if there are strings inside
  your NAPTR lookups.

  \item If there exist multiple records with the same weight and order as
  a result of your query, the function will RANDOMLY select a single
  NAPTR from those equal results.

  \item Currently, the function ignores the settings in enum.conf as the
  search zone name is now specified within the function, and the H323
  driver can be chosen by the user via the dialplan.  There were no
  other values in this file, and so it becomes deprecated.

  \item The function will digest and return NAPTRs which use older
  (deprecated) style, reversed method strings such as "sip+E2U"
  instead of the more modern "E2U+sip"

  \item There is no provision for multi-part methods at this time.  If
  there are multiple NAPTRs with (as an example) a method of
  "E2U+voice:sip" and then another NAPTR in the same DNS record with a
  method of ""E2U+sip", the system will treat these both as method
  "sip" and they will be separate records from the perspective of the
  function.  Of course, if both records point to the same URI and have
  equal priority/weight (as is often the case) then this will cause no
  serious difficulty, but it bears mentioning.

  \item ISN (ITAD Subscriber Number) usage:  If the search number is of
  the form ABC*DEF (where ABC and DEF are at least one numeric digit)
  then perform an ISN-style lookup where the lookup is manipulated to
  C.B.A.DEF.domain.tld (all other settings and options apply.)  See
  \url{http://www.freenum.org/} for more details on ISN lookups.  In the
  unlikely event you wish to avoid ISN re-writes, put an "n" as the
  first digit of the search string - the "n" will be ignored for the search.
\end{itemize}

\subsection{Some more Examples}

All examples below except where noted use "e164.arpa" as the
referenced domain, which is the default domain name for ENUMLOOKUP.
All numbers are assumed to not have a leading "+" as dialed by the
inbound channel, so that character is added where necessary during
ENUMLOOKUP function calls.

\begin{astlisting}
\begin{verbatim}
; example 1
;
; Assumes North American international dialing (011) prefix.
; Look up the first SIP result and send the call there, otherwise
;  send the call out a PRI.  This is the most simple possible
;  ENUM example, but only uses the first SIP reply in the list of
;  NAPTR(s).
;
exten => _011.,1,Set(enumresult=${ENUMLOOKUP(+${EXTEN:3})})
exten => _011.,n,Dial(SIP/${enumresult})
exten => _011.,n,Dial(DAHDI/g1/${EXTEN})
;
; end example 1

; example 2
;
; Assumes North American international dialing (011) prefix.
; Check to see if there are multiple SIP NAPTRs returned by
;  the lookup, and dial each in order.  If none work (or none
;  exist) then send the call out a PRI, group 1.
;
exten => _011.,1,Set(sipcount=${ENUMLOOKUP(${EXTEN:3},sip,c)}|counter=0)
exten => _011.,n,While($["${counter}"<"${sipcount}"])
exten => _011.,n,Set(counter=$[${counter}+1])
exten => _011.,n,Dial(SIP/${ENUMLOOKUP(+${EXTEN:3},sip,,${counter})})
exten => _011.,n,EndWhile
exten => _011.,n,Dial(DAHDI/g1/${EXTEN})
;
; end example 2

; example 3
;
; This example expects an ${EXTEN} that is an e.164 number (like
;  14102241145 or 437203001721)
; Search through e164.arpa and then also search through e164.org
;  to see if there are any valid SIP or IAX termination capabilities.
;  If none, send call out via DAHDI channel 1.
;
; Start first with e164.arpa zone...
;
exten => _X.,1,Set(sipcount=${ENUMLOOKUP(+${EXTEN},sip,c)}|counter=0)
exten => _X.,2,GotoIf($["${counter}"<"${sipcount}"]?3:6)
exten => _X.,3,Set(counter=$[${counter}+1])
exten => _X.,4,Dial(SIP/${ENUMLOOKUP(+${EXTEN},sip,,${counter})})
exten => _X.,5,GotoIf($["${counter}"<"${sipcount}"]?3:6)
;
exten => _X.,6,Set(iaxcount=${ENUMLOOKUP(+${EXTEN},iax2,c)}|counter=0)
exten => _X.,7,GotoIf($["${counter}"<"${iaxcount}"]?8:11)
exten => _X.,8,Set(counter=$[${counter}+1])
exten => _X.,9,Dial(IAX2/${ENUMLOOKUP(+${EXTEN},iax2,,${counter})})
exten => _X.,10,GotoIf($["${counter}"<"${iaxcount}"]?8:11)
;
exten => _X.,11,NoOp("No valid entries in e164.arpa for ${EXTEN} - checking in e164.org")
;
; ...then also try e164.org, and look for SIP and IAX NAPTRs...
;
exten => _X.,12,Set(sipcount=${ENUMLOOKUP(+${EXTEN},sip,c,,e164.org)}|counter=0)
exten => _X.,13,GotoIf($["${counter}"<"${sipcount}"]?14:17)
exten => _X.,14,Set(counter=$[${counter}+1])
exten => _X.,15,Dial(SIP/${ENUMLOOKUP(+${EXTEN},sip,,${counter},e164.org)})
exten => _X.,16,GotoIf($["${counter}"<"${sipcount}"]?14:17)
;
exten => _X.,17,Set(iaxcount=${ENUMLOOKUP(+${EXTEN},iax2,c,,e164.org)}|counter=0)
exten => _X.,18,GotoIf($["${counter}"<"${iaxcount}"]?19:22)
exten => _X.,19,Set(counter=$[${counter}+1])
exten => _X.,20,Dial(IAX2/${ENUMLOOKUP(+${EXTEN},iax2,,${counter},e164.org)})
exten => _X.,21,GotoIf($["${counter}"<"${iaxcount}"]?19:22)
;
; ...then send out PRI.
;
exten => _X.,22,NoOp("No valid entries in e164.org for ${EXTEN} - sending out via DAHDI")
exten => _X.,23,Dial(DAHDI/g1/${EXTEN})
;
; end example 3

\end{verbatim}
\end{astlisting}
