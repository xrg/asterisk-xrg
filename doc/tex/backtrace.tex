This document is intended to provide information on how to obtain the
backtraces required on the asterisk bug tracker, available at
\url{http://bugs.digium.com}. The information is required by developers to
help fix problem with bugs of any kind. Backtraces provide information
about what was wrong when a program crashed; in our case,
Asterisk. There are two kind of backtraces (aka 'bt') which are
useful: bt and bt full.

First of all, when you start Asterisk, you MUST start it with option
-g. This tells Asterisk to produce a core file if it crashes.

If you start Asterisk with the safe\_asterisk script, it automatically
starts using the option -g.

If you're not sure if Asterisk is running with the -g option, type the
following command in your shell:

\begin{astlisting}
\begin{verbatim}
debian:/tmp# ps aux | grep asterisk
root     17832  0.0  1.2   2348   788 pts/1    S    Aug12   0:00 /bin/sh /usr/sbin/safe_asterisk
root     26686  0.0  2.8  15544  1744 pts/1    S    Aug13   0:02 asterisk -vvvg -c
[...]
\end{verbatim}
\end{astlisting}

The interesting information is located in the last column.

Second, your copy of Asterisk must have been built without
optimization or the backtrace will be (nearly) unusable. This can be
done by selecting the 'DONT\_OPTIMIZE' option in the Compiler Flags
submenu in the 'make menuselect' tree before building Asterisk.

After Asterisk crashes, a core file will be "dumped" in your \path{/tmp/}
directory. To make sure it's really there, you can just type the
following command in your shell:

\begin{astlisting}
\begin{verbatim}
debian:/tmp# ls -l /tmp/core.*
-rw-------  1 root root 10592256 Aug 12 19:40 /tmp/core.26252
-rw-------  1 root root  9924608 Aug 12 20:12 /tmp/core.26340
-rw-------  1 root root 10862592 Aug 12 20:14 /tmp/core.26374
-rw-------  1 root root  9105408 Aug 12 20:19 /tmp/core.26426
-rw-------  1 root root  9441280 Aug 12 20:20 /tmp/core.26462
-rw-------  1 root root  8331264 Aug 13 00:32 /tmp/core.26647
debian:/tmp#
\end{verbatim}
\end{astlisting}

In the event that there are multiple core files present (as in the
above example), it is important to look at the file timestamps in
order to determine which one you really intend to look at.

Now that we've verified the core file has been written to disk, the
final part is to extract 'bt' from the core file. Core files are
pretty big, don't be scared, it's normal.

\textbf{NOTE: Don't attach core files on the bug tracker, we only need the bt and bt full.}

For extraction, we use a really nice tool, called gdb. To verify that
you have gdb installed on your system:

\begin{astlisting}
\begin{verbatim}
debian:/tmp# gdb -v
GNU gdb 6.3-debian
Copyright 2004 Free Software Foundation, Inc.
GDB is free software, covered by the GNU General Public License, and you are
welcome to change it and/or distribute copies of it under certain conditions.
Type "show copying" to see the conditions.
There is absolutely no warranty for GDB.  Type "show warranty" for details.
This GDB was configured as "i386-linux".
debian:/tmp#
\end{verbatim}
\end{astlisting}

Which is great, we can continue. If you don't have gdb installed, go install gdb.

Now load the core file in gdb, as follows:

\begin{astlisting}
\begin{verbatim}
debian:/tmp# gdb asterisk /tmp/core.26252
[...]
(You would see a lot of output here.)
[...]
Reading symbols from /usr/lib/asterisk/modules/app_externalivr.so...done.
Loaded symbols for /usr/lib/asterisk/modules/app_externalivr.so
#0  0x29b45d7e in ?? ()
(gdb)
\end{verbatim}
\end{astlisting}

Now at the gdb prompt, type: bt
You would see output similar to:

\begin{astlisting}
\begin{verbatim}
(gdb) bt
#0  0x29b45d7e in ?? ()
#1  0x08180bf8 in ?? ()
#2  0xbcdffa58 in ?? ()
#3  0x08180bf8 in ?? ()
#4  0xbcdffa60 in ?? ()
#5  0x08180bf8 in ?? ()
#6  0x180bf894 in ?? ()
#7  0x0bf80008 in ?? ()
#8  0x180b0818 in ?? ()
#9  0x08068008 in ast_stopstream (tmp=0x40758d38) at file.c:180
#10 0x000000a0 in ?? ()
#11 0x000000a0 in ?? ()
#12 0x00000000 in ?? ()
#13 0x407513c3 in confcall_careful_stream (conf=0x8180bf8, filename=0x8181de8 "DAHDI/pseudo-1324221520") at app_meetme.c:262
#14 0x40751332 in streamconfthread (args=0x8180bf8) at app_meetme.c:1965
#15 0xbcdffbe0 in ?? ()
#16 0x40028e51 in pthread_start_thread () from /lib/libpthread.so.0
#17 0x401ec92a in clone () from /lib/libc.so.6
(gdb)
\end{verbatim}
\end{astlisting}

The bt's output is the information that we need on the bug tracker.

\begin{astlisting}
\begin{verbatim}
Now do a bt full as follows:
(gdb) bt full
#0  0x29b45d7e in ?? ()
No symbol table info available.
#1  0x08180bf8 in ?? ()
No symbol table info available.
#2  0xbcdffa58 in ?? ()
No symbol table info available.
#3  0x08180bf8 in ?? ()
No symbol table info available.
#4  0xbcdffa60 in ?? ()
No symbol table info available.
#5  0x08180bf8 in ?? ()
No symbol table info available.
#6  0x180bf894 in ?? ()
No symbol table info available.
#7  0x0bf80008 in ?? ()
No symbol table info available.
#8  0x180b0818 in ?? ()
No symbol table info available.
#9  0x08068008 in ast_stopstream (tmp=0x40758d38) at file.c:180
No locals.
#10 0x000000a0 in ?? ()
No symbol table info available.
#11 0x000000a0 in ?? ()
No symbol table info available.
#12 0x00000000 in ?? ()
No symbol table info available.
#13 0x407513c3 in confcall_careful_stream (conf=0x8180bf8, filename=0x8181de8 "DAHDI/pseudo-1324221520") at app_meetme.c:262
        f = (struct ast_frame *) 0x8180bf8
        trans = (struct ast_trans_pvt *) 0x0
#14 0x40751332 in streamconfthread (args=0x8180bf8) at app_meetme.c:1965
No locals.
#15 0xbcdffbe0 in ?? ()
No symbol table info available.
#16 0x40028e51 in pthread_start_thread () from /lib/libpthread.so.0
No symbol table info available.
#17 0x401ec92a in clone () from /lib/libc.so.6
No symbol table info available.
(gdb)
\end{verbatim}
\end{astlisting}

We also need gdb's output. That output gives more details compared to
the simple "bt". So we recommend that you use bt full instead of bt.
But, if you could include both, we appreciate that.

The final "extraction" would be to know all traces by all
threads. Even if asterisk runs on the same thread for each call, it
could have created some new threads.

To make sure we have the correct information, just do:
(gdb) thread apply all bt

\begin{astlisting}
\begin{verbatim}
Thread 1 (process 26252):
#0  0x29b45d7e in ?? ()
#1  0x08180bf8 in ?? ()
#2  0xbcdffa58 in ?? ()
#3  0x08180bf8 in ?? ()
#4  0xbcdffa60 in ?? ()
#5  0x08180bf8 in ?? ()
#6  0x180bf894 in ?? ()
#7  0x0bf80008 in ?? ()
#8  0x180b0818 in ?? ()
#9  0x08068008 in ast_stopstream (tmp=0x40758d38) at file.c:180
#10 0x000000a0 in ?? ()
#11 0x000000a0 in ?? ()
#12 0x00000000 in ?? ()
#13 0x407513c3 in confcall_careful_stream (conf=0x8180bf8, filename=0x8181de8 "DAHDI/pseudo-1324221520") at app_meetme.c:262
#14 0x40751332 in streamconfthread (args=0x8180bf8) at app_meetme.c:1965
#15 0xbcdffbe0 in ?? ()
#16 0x40028e51 in pthread_start_thread () from /lib/libpthread.so.0
#17 0x401ec92a in clone () from /lib/libc.so.6
(gdb)
\end{verbatim}
\end{astlisting}

That output tells us crucial information about each thread.

Now, just create an output.txt file and dump your "bt full"
(and/or "bt") ALONG WITH "thread apply all bt" into it.

Note: Please ATTACH your output, DO NOT paste it as a note.

And you're ready for upload on the bug tracker.

If you have questions or comments regarding this documentation, feel
free to pass by the \#asterisk-bugs channel on irc.freenode.net.

