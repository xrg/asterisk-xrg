% This file is automatically generated by the "manager dump actiondocs" CLI command.  Any manual edits will be lost.
\section{AbsoluteTimeout}
\subsection{Synopsis}
\begin{verbatim}
Set Absolute Timeout
\end{verbatim}
\subsection{Authority}
\begin{verbatim}
call,all
\end{verbatim}
\subsection{Description}
\begin{verbatim}
Description: Hangup a channel after a certain time.
Variables: (Names marked with * are required)
	*Channel: Channel name to hangup
	*Timeout: Maximum duration of the call (sec)
Acknowledges set time with 'Timeout Set' message

\end{verbatim}


\section{AgentLogoff}
\subsection{Synopsis}
\begin{verbatim}
Sets an agent as no longer logged in
\end{verbatim}
\subsection{Authority}
\begin{verbatim}
agent,all
\end{verbatim}
\subsection{Description}
\begin{verbatim}
Description: Sets an agent as no longer logged in.
Variables: (Names marked with * are required)
	*Agent: Agent ID of the agent to log off
	Soft: Set to 'true' to not hangup existing calls

\end{verbatim}


\section{Agents}
\subsection{Synopsis}
\begin{verbatim}
Lists agents and their status
\end{verbatim}
\subsection{Authority}
\begin{verbatim}
agent,all
\end{verbatim}
\subsection{Description}
\begin{verbatim}
Description: Will list info about all possible agents.
Variables: NONE

\end{verbatim}


\section{Bridge}
\subsection{Synopsis}
\begin{verbatim}
Bridge two channels already in the PBX
\end{verbatim}
\subsection{Authority}
\begin{verbatim}
command,all
\end{verbatim}
\subsection{Description}
\begin{verbatim}
Description: Bridge together two channels already in the PBX
Variables: ( Headers marked with * are required )
   *Channel1: Channel to Bridge to Channel2
   *Channel2: Channel to Bridge to Channel1
        Tone: (Yes|No) Play courtesy tone to Channel 2


\end{verbatim}


\section{Challenge}
\subsection{Synopsis}
\begin{verbatim}
Generate Challenge for MD5 Auth
\end{verbatim}
\subsection{Authority}
\begin{verbatim}
<none>
\end{verbatim}
\subsection{Description}
\begin{verbatim}
(null)
\end{verbatim}


\section{ChangeMonitor}
\subsection{Synopsis}
\begin{verbatim}
Change monitoring filename of a channel
\end{verbatim}
\subsection{Authority}
\begin{verbatim}
call,all
\end{verbatim}
\subsection{Description}
\begin{verbatim}
Description: The 'ChangeMonitor' action may be used to change the file
  started by a previous 'Monitor' action.  The following parameters may
  be used to control this:
  Channel     - Required.  Used to specify the channel to record.
  File        - Required.  Is the new name of the file created in the
                monitor spool directory.

\end{verbatim}


\section{Command}
\subsection{Synopsis}
\begin{verbatim}
Execute Asterisk CLI Command
\end{verbatim}
\subsection{Authority}
\begin{verbatim}
command,all
\end{verbatim}
\subsection{Description}
\begin{verbatim}
Description: Run a CLI command.
Variables: (Names marked with * are required)
	*Command: Asterisk CLI command to run
	ActionID: Optional Action id for message matching.

\end{verbatim}


\section{CoreSettings}
\subsection{Synopsis}
\begin{verbatim}
Show PBX core settings (version etc)
\end{verbatim}
\subsection{Authority}
\begin{verbatim}
system,all
\end{verbatim}
\subsection{Description}
\begin{verbatim}
Description: Query for Core PBX settings.
Variables: (Names marked with * are optional)
       *ActionID: ActionID of this transaction

\end{verbatim}


\section{CoreStatus}
\subsection{Synopsis}
\begin{verbatim}
Show PBX core status variables
\end{verbatim}
\subsection{Authority}
\begin{verbatim}
system,all
\end{verbatim}
\subsection{Description}
\begin{verbatim}
Description: Query for Core PBX status.
Variables: (Names marked with * are optional)
       *ActionID: ActionID of this transaction

\end{verbatim}


\section{DBDel}
\subsection{Synopsis}
\begin{verbatim}
Delete DB Entry
\end{verbatim}
\subsection{Authority}
\begin{verbatim}
system,all
\end{verbatim}
\subsection{Description}
\begin{verbatim}
(null)
\end{verbatim}


\section{DBDelTree}
\subsection{Synopsis}
\begin{verbatim}
Delete DB Tree
\end{verbatim}
\subsection{Authority}
\begin{verbatim}
system,all
\end{verbatim}
\subsection{Description}
\begin{verbatim}
(null)
\end{verbatim}


\section{DBGet}
\subsection{Synopsis}
\begin{verbatim}
Get DB Entry
\end{verbatim}
\subsection{Authority}
\begin{verbatim}
system,all
\end{verbatim}
\subsection{Description}
\begin{verbatim}
(null)
\end{verbatim}


\section{DBPut}
\subsection{Synopsis}
\begin{verbatim}
Put DB Entry
\end{verbatim}
\subsection{Authority}
\begin{verbatim}
system,all
\end{verbatim}
\subsection{Description}
\begin{verbatim}
(null)
\end{verbatim}


\section{Events}
\subsection{Synopsis}
\begin{verbatim}
Control Event Flow
\end{verbatim}
\subsection{Authority}
\begin{verbatim}
<none>
\end{verbatim}
\subsection{Description}
\begin{verbatim}
Description: Enable/Disable sending of events to this manager
  client.
Variables:
	EventMask: 'on' if all events should be sent,
		'off' if no events should be sent,
		'system,call,log' to select which flags events should have to be sent.

\end{verbatim}


\section{ExtensionState}
\subsection{Synopsis}
\begin{verbatim}
Check Extension Status
\end{verbatim}
\subsection{Authority}
\begin{verbatim}
call,all
\end{verbatim}
\subsection{Description}
\begin{verbatim}
Description: Report the extension state for given extension.
  If the extension has a hint, will use devicestate to check
  the status of the device connected to the extension.
Variables: (Names marked with * are required)
	*Exten: Extension to check state on
	*Context: Context for extension
	ActionId: Optional ID for this transaction
Will return an "Extension Status" message.
The response will include the hint for the extension and the status.

\end{verbatim}


\section{GetConfig}
\subsection{Synopsis}
\begin{verbatim}
Retrieve configuration
\end{verbatim}
\subsection{Authority}
\begin{verbatim}
config,all
\end{verbatim}
\subsection{Description}
\begin{verbatim}
Description: A 'GetConfig' action will dump the contents of a configuration
file by category and contents.
Variables:
   Filename: Configuration filename (e.g. foo.conf)

\end{verbatim}


\section{GetConfigJSON}
\subsection{Synopsis}
\begin{verbatim}
Retrieve configuration (JSON format)
\end{verbatim}
\subsection{Authority}
\begin{verbatim}
config,all
\end{verbatim}
\subsection{Description}
\begin{verbatim}
Description: A 'GetConfigJSON' action will dump the contents of a configuration
file by category and contents in JSON format.  This only makes sense to be used
using rawman over the HTTP interface.
Variables:
   Filename: Configuration filename (e.g. foo.conf)

\end{verbatim}


\section{Getvar}
\subsection{Synopsis}
\begin{verbatim}
Gets a Channel Variable
\end{verbatim}
\subsection{Authority}
\begin{verbatim}
call,all
\end{verbatim}
\subsection{Description}
\begin{verbatim}
Description: Get the value of a global or local channel variable.
Variables: (Names marked with * are required)
	Channel: Channel to read variable from
	*Variable: Variable name
	ActionID: Optional Action id for message matching.

\end{verbatim}


\section{Hangup}
\subsection{Synopsis}
\begin{verbatim}
Hangup Channel
\end{verbatim}
\subsection{Authority}
\begin{verbatim}
call,all
\end{verbatim}
\subsection{Description}
\begin{verbatim}
Description: Hangup a channel
Variables: 
	Channel: The channel name to be hungup

\end{verbatim}


\section{IAXnetstats}
\subsection{Synopsis}
\begin{verbatim}
Show IAX Netstats
\end{verbatim}
\subsection{Authority}
\begin{verbatim}
<none>
\end{verbatim}
\subsection{Description}
\begin{verbatim}
(null)
\end{verbatim}


\section{IAXpeers}
\subsection{Synopsis}
\begin{verbatim}
List IAX Peers
\end{verbatim}
\subsection{Authority}
\begin{verbatim}
<none>
\end{verbatim}
\subsection{Description}
\begin{verbatim}
(null)
\end{verbatim}


\section{JabberSend}
\subsection{Synopsis}
\begin{verbatim}
Sends a message to a Jabber Client
\end{verbatim}
\subsection{Authority}
\begin{verbatim}
system,all
\end{verbatim}
\subsection{Description}
\begin{verbatim}
Description: Sends a message to a Jabber Client.
Variables: 
  Jabber:	Client or transport Asterisk uses to connect to JABBER.
  ScreenName:	User Name to message.
  Message:	Message to be sent to the buddy

\end{verbatim}


\section{ListCommands}
\subsection{Synopsis}
\begin{verbatim}
List available manager commands
\end{verbatim}
\subsection{Authority}
\begin{verbatim}
<none>
\end{verbatim}
\subsection{Description}
\begin{verbatim}
Description: Returns the action name and synopsis for every
  action that is available to the user
Variables: NONE

\end{verbatim}


\section{Login}
\subsection{Synopsis}
\begin{verbatim}
Login Manager
\end{verbatim}
\subsection{Authority}
\begin{verbatim}
<none>
\end{verbatim}
\subsection{Description}
\begin{verbatim}
(null)
\end{verbatim}


\section{Logoff}
\subsection{Synopsis}
\begin{verbatim}
Logoff Manager
\end{verbatim}
\subsection{Authority}
\begin{verbatim}
<none>
\end{verbatim}
\subsection{Description}
\begin{verbatim}
Description: Logoff this manager session
Variables: NONE

\end{verbatim}


\section{MailboxCount}
\subsection{Synopsis}
\begin{verbatim}
Check Mailbox Message Count
\end{verbatim}
\subsection{Authority}
\begin{verbatim}
call,all
\end{verbatim}
\subsection{Description}
\begin{verbatim}
Description: Checks a voicemail account for new messages.
Variables: (Names marked with * are required)
	*Mailbox: Full mailbox ID <mailbox>@<vm-context>
	ActionID: Optional ActionID for message matching.
Returns number of new and old messages.
	Message: Mailbox Message Count
	Mailbox: <mailboxid>
	NewMessages: <count>
	OldMessages: <count>


\end{verbatim}


\section{MailboxStatus}
\subsection{Synopsis}
\begin{verbatim}
Check Mailbox
\end{verbatim}
\subsection{Authority}
\begin{verbatim}
call,all
\end{verbatim}
\subsection{Description}
\begin{verbatim}
Description: Checks a voicemail account for status.
Variables: (Names marked with * are required)
	*Mailbox: Full mailbox ID <mailbox>@<vm-context>
	ActionID: Optional ActionID for message matching.
Returns number of messages.
	Message: Mailbox Status
	Mailbox: <mailboxid>
	Waiting: <count>


\end{verbatim}


\section{MeetmeMute}
\subsection{Synopsis}
\begin{verbatim}
Mute a Meetme user
\end{verbatim}
\subsection{Authority}
\begin{verbatim}
call,all
\end{verbatim}
\subsection{Description}
\begin{verbatim}
(null)
\end{verbatim}


\section{MeetmeUnmute}
\subsection{Synopsis}
\begin{verbatim}
Unmute a Meetme user
\end{verbatim}
\subsection{Authority}
\begin{verbatim}
call,all
\end{verbatim}
\subsection{Description}
\begin{verbatim}
(null)
\end{verbatim}


\section{Monitor}
\subsection{Synopsis}
\begin{verbatim}
Monitor a channel
\end{verbatim}
\subsection{Authority}
\begin{verbatim}
call,all
\end{verbatim}
\subsection{Description}
\begin{verbatim}
Description: The 'Monitor' action may be used to record the audio on a
  specified channel.  The following parameters may be used to control
  this:
  Channel     - Required.  Used to specify the channel to record.
  File        - Optional.  Is the name of the file created in the
                monitor spool directory.  Defaults to the same name
                as the channel (with slashes replaced with dashes).
  Format      - Optional.  Is the audio recording format.  Defaults
                to "wav".
  Mix         - Optional.  Boolean parameter as to whether to mix
                the input and output channels together after the
                recording is finished.

\end{verbatim}


\section{Originate}
\subsection{Synopsis}
\begin{verbatim}
Originate Call
\end{verbatim}
\subsection{Authority}
\begin{verbatim}
call,all
\end{verbatim}
\subsection{Description}
\begin{verbatim}
Description: Generates an outgoing call to a Extension/Context/Priority or
  Application/Data
Variables: (Names marked with * are required)
	*Channel: Channel name to call
	Exten: Extension to use (requires 'Context' and 'Priority')
	Context: Context to use (requires 'Exten' and 'Priority')
	Priority: Priority to use (requires 'Exten' and 'Context')
	Application: Application to use
	Data: Data to use (requires 'Application')
	Timeout: How long to wait for call to be answered (in ms)
	CallerID: Caller ID to be set on the outgoing channel
	Variable: Channel variable to set, multiple Variable: headers are allowed
	Account: Account code
	Async: Set to 'true' for fast origination

\end{verbatim}


\section{Park}
\subsection{Synopsis}
\begin{verbatim}
Park a channel
\end{verbatim}
\subsection{Authority}
\begin{verbatim}
call,all
\end{verbatim}
\subsection{Description}
\begin{verbatim}
Description: Park a channel.
Variables: (Names marked with * are required)
	*Channel: Channel name to park
	*Channel2: Channel to announce park info to (and return to if timeout)
	Timeout: Number of milliseconds to wait before callback.

\end{verbatim}


\section{ParkedCalls}
\subsection{Synopsis}
\begin{verbatim}
List parked calls
\end{verbatim}
\subsection{Authority}
\begin{verbatim}
<none>
\end{verbatim}
\subsection{Description}
\begin{verbatim}
(null)
\end{verbatim}


\section{PauseMonitor}
\subsection{Synopsis}
\begin{verbatim}
Pause monitoring of a channel
\end{verbatim}
\subsection{Authority}
\begin{verbatim}
call,all
\end{verbatim}
\subsection{Description}
\begin{verbatim}
Description: The 'PauseMonitor' action may be used to temporarily stop the
 recording of a channel.  The following parameters may
 be used to control this:
  Channel     - Required.  Used to specify the channel to record.

\end{verbatim}


\section{Ping}
\subsection{Synopsis}
\begin{verbatim}
Keepalive command
\end{verbatim}
\subsection{Authority}
\begin{verbatim}
<none>
\end{verbatim}
\subsection{Description}
\begin{verbatim}
Description: A 'Ping' action will ellicit a 'Pong' response.  Used to keep the
  manager connection open.
Variables: NONE

\end{verbatim}


\section{PlayDTMF}
\subsection{Synopsis}
\begin{verbatim}
Play DTMF signal on a specific channel.
\end{verbatim}
\subsection{Authority}
\begin{verbatim}
call,all
\end{verbatim}
\subsection{Description}
\begin{verbatim}
Description: Plays a dtmf digit on the specified channel.
Variables: (all are required)
	Channel: Channel name to send digit to
	Digit: The dtmf digit to play

\end{verbatim}


\section{QueueAdd}
\subsection{Synopsis}
\begin{verbatim}
Add interface to queue.
\end{verbatim}
\subsection{Authority}
\begin{verbatim}
agent,all
\end{verbatim}
\subsection{Description}
\begin{verbatim}
(null)
\end{verbatim}


\section{QueueLog}
\subsection{Synopsis}
\begin{verbatim}
Adds custom entry in queue_log
\end{verbatim}
\subsection{Authority}
\begin{verbatim}
agent,all
\end{verbatim}
\subsection{Description}
\begin{verbatim}
(null)
\end{verbatim}


\section{QueuePause}
\subsection{Synopsis}
\begin{verbatim}
Makes a queue member temporarily unavailable
\end{verbatim}
\subsection{Authority}
\begin{verbatim}
agent,all
\end{verbatim}
\subsection{Description}
\begin{verbatim}
(null)
\end{verbatim}


\section{QueueRemove}
\subsection{Synopsis}
\begin{verbatim}
Remove interface from queue.
\end{verbatim}
\subsection{Authority}
\begin{verbatim}
agent,all
\end{verbatim}
\subsection{Description}
\begin{verbatim}
(null)
\end{verbatim}


\section{Queues}
\subsection{Synopsis}
\begin{verbatim}
Queues
\end{verbatim}
\subsection{Authority}
\begin{verbatim}
<none>
\end{verbatim}
\subsection{Description}
\begin{verbatim}
(null)
\end{verbatim}


\section{QueueStatus}
\subsection{Synopsis}
\begin{verbatim}
Queue Status
\end{verbatim}
\subsection{Authority}
\begin{verbatim}
<none>
\end{verbatim}
\subsection{Description}
\begin{verbatim}
(null)
\end{verbatim}


\section{QueueSummary}
\subsection{Synopsis}
\begin{verbatim}
Queue Summary
\end{verbatim}
\subsection{Authority}
\begin{verbatim}
<none>
\end{verbatim}
\subsection{Description}
\begin{verbatim}
(null)
\end{verbatim}


\section{Redirect}
\subsection{Synopsis}
\begin{verbatim}
Redirect (transfer) a call
\end{verbatim}
\subsection{Authority}
\begin{verbatim}
call,all
\end{verbatim}
\subsection{Description}
\begin{verbatim}
Description: Redirect (transfer) a call.
Variables: (Names marked with * are required)
	*Channel: Channel to redirect
	ExtraChannel: Second call leg to transfer (optional)
	*Exten: Extension to transfer to
	*Context: Context to transfer to
	*Priority: Priority to transfer to
	ActionID: Optional Action id for message matching.

\end{verbatim}


\section{SendText}
\subsection{Synopsis}
\begin{verbatim}
Send text message to channel
\end{verbatim}
\subsection{Authority}
\begin{verbatim}
call,all
\end{verbatim}
\subsection{Description}
\begin{verbatim}
Description: Sends A Text Message while in a call.
Variables: (Names marked with * are required)
       *Channel: Channel to send message to
       *Message: Message to send
       ActionID: Optional Action id for message matching.

\end{verbatim}


\section{Setvar}
\subsection{Synopsis}
\begin{verbatim}
Set Channel Variable
\end{verbatim}
\subsection{Authority}
\begin{verbatim}
call,all
\end{verbatim}
\subsection{Description}
\begin{verbatim}
Description: Set a global or local channel variable.
Variables: (Names marked with * are required)
	Channel: Channel to set variable for
	*Variable: Variable name
	*Value: Value

\end{verbatim}


\section{ShowDialPlan}
\subsection{Synopsis}
\begin{verbatim}
List dialplan
\end{verbatim}
\subsection{Authority}
\begin{verbatim}
config,all
\end{verbatim}
\subsection{Description}
\begin{verbatim}
Description: Show dialplan contexts and extensions.
Be aware that showing the full dialplan may take a lot of capacity
Variables: 
 ActionID: <id>		Action ID for this AMI transaction (optional)
 Extension: <extension>	Extension (Optional)
 Context: <context>		Context (Optional)


\end{verbatim}


\section{SIPpeers}
\subsection{Synopsis}
\begin{verbatim}
List SIP peers (text format)
\end{verbatim}
\subsection{Authority}
\begin{verbatim}
system,all
\end{verbatim}
\subsection{Description}
\begin{verbatim}
Description: Lists SIP peers in text format with details on current status.
Peerlist will follow as separate events, followed by a final event called
PeerlistComplete.
Variables: 
  ActionID: <id>	Action ID for this transaction. Will be returned.

\end{verbatim}


\section{SIPshowpeer}
\subsection{Synopsis}
\begin{verbatim}
Show SIP peer (text format)
\end{verbatim}
\subsection{Authority}
\begin{verbatim}
system,all
\end{verbatim}
\subsection{Description}
\begin{verbatim}
Description: Show one SIP peer with details on current status.
Variables: 
  Peer: <name>           The peer name you want to check.
  ActionID: <id>	  Optional action ID for this AMI transaction.

\end{verbatim}


\section{Status}
\subsection{Synopsis}
\begin{verbatim}
Lists channel status
\end{verbatim}
\subsection{Authority}
\begin{verbatim}
call,all
\end{verbatim}
\subsection{Description}
\begin{verbatim}
(null)
\end{verbatim}


\section{StopMonitor}
\subsection{Synopsis}
\begin{verbatim}
Stop monitoring a channel
\end{verbatim}
\subsection{Authority}
\begin{verbatim}
call,all
\end{verbatim}
\subsection{Description}
\begin{verbatim}
Description: The 'StopMonitor' action may be used to end a previously
  started 'Monitor' action.  The only parameter is 'Channel', the name
  of the channel monitored.

\end{verbatim}


\section{UnpauseMonitor}
\subsection{Synopsis}
\begin{verbatim}
Unpause monitoring of a channel
\end{verbatim}
\subsection{Authority}
\begin{verbatim}
call,all
\end{verbatim}
\subsection{Description}
\begin{verbatim}
Description: The 'UnpauseMonitor' action may be used to re-enable recording
  of a channel after calling PauseMonitor.  The following parameters may
  be used to control this:
  Channel     - Required.  Used to specify the channel to record.

\end{verbatim}


\section{UpdateConfig}
\subsection{Synopsis}
\begin{verbatim}
Update basic configuration
\end{verbatim}
\subsection{Authority}
\begin{verbatim}
config,all
\end{verbatim}
\subsection{Description}
\begin{verbatim}
Description: A 'UpdateConfig' action will dump the contents of a configuration
file by category and contents.
Variables (X's represent 6 digit number beginning with 000000):
   SrcFilename:   Configuration filename to read(e.g. foo.conf)
   DstFilename:   Configuration filename to write(e.g. foo.conf)
   Reload:        Whether or not a reload should take place (or name of specific module)
   Action-XXXXXX: Action to Take (NewCat,RenameCat,DelCat,Update,Delete,Append)
   Cat-XXXXXX:    Category to operate on
   Var-XXXXXX:    Variable to work on
   Value-XXXXXX:  Value to work on
   Match-XXXXXX:  Extra match required to match line

\end{verbatim}


\section{UserEvent}
\subsection{Synopsis}
\begin{verbatim}
Send an arbitrary event
\end{verbatim}
\subsection{Authority}
\begin{verbatim}
user,all
\end{verbatim}
\subsection{Description}
\begin{verbatim}
Description: Send an event to manager sessions.
Variables: (Names marked with * are required)
       *UserEvent: EventStringToSend
       Header1: Content1
       HeaderN: ContentN

\end{verbatim}


\section{VoicemailUsersList}
\subsection{Synopsis}
\begin{verbatim}
List All Voicemail User Information
\end{verbatim}
\subsection{Authority}
\begin{verbatim}
call,all
\end{verbatim}
\subsection{Description}
\begin{verbatim}
(null)
\end{verbatim}


\section{WaitEvent}
\subsection{Synopsis}
\begin{verbatim}
Wait for an event to occur
\end{verbatim}
\subsection{Authority}
\begin{verbatim}
<none>
\end{verbatim}
\subsection{Description}
\begin{verbatim}
Description: A 'WaitEvent' action will ellicit a 'Success' response.  Whenever
a manager event is queued.  Once WaitEvent has been called on an HTTP manager
session, events will be generated and queued.
Variables: 
   Timeout: Maximum time (in seconds) to wait for events, -1 means forever.

\end{verbatim}


\section{ZapDialOffhook}
\subsection{Synopsis}
\begin{verbatim}
Dial over Zap channel while offhook
\end{verbatim}
\subsection{Authority}
\begin{verbatim}
<none>
\end{verbatim}
\subsection{Description}
\begin{verbatim}
(null)
\end{verbatim}


\section{ZapDNDoff}
\subsection{Synopsis}
\begin{verbatim}
Toggle Zap channel Do Not Disturb status OFF
\end{verbatim}
\subsection{Authority}
\begin{verbatim}
<none>
\end{verbatim}
\subsection{Description}
\begin{verbatim}
(null)
\end{verbatim}


\section{ZapDNDon}
\subsection{Synopsis}
\begin{verbatim}
Toggle Zap channel Do Not Disturb status ON
\end{verbatim}
\subsection{Authority}
\begin{verbatim}
<none>
\end{verbatim}
\subsection{Description}
\begin{verbatim}
(null)
\end{verbatim}


\section{ZapHangup}
\subsection{Synopsis}
\begin{verbatim}
Hangup Zap Channel
\end{verbatim}
\subsection{Authority}
\begin{verbatim}
<none>
\end{verbatim}
\subsection{Description}
\begin{verbatim}
(null)
\end{verbatim}


\section{ZapRestart}
\subsection{Synopsis}
\begin{verbatim}
Fully Restart zaptel channels (terminates calls)
\end{verbatim}
\subsection{Authority}
\begin{verbatim}
<none>
\end{verbatim}
\subsection{Description}
\begin{verbatim}
(null)
\end{verbatim}


\section{ZapShowChannels}
\subsection{Synopsis}
\begin{verbatim}
Show status zapata channels
\end{verbatim}
\subsection{Authority}
\begin{verbatim}
<none>
\end{verbatim}
\subsection{Description}
\begin{verbatim}
(null)
\end{verbatim}


\section{ZapTransfer}
\subsection{Synopsis}
\begin{verbatim}
Transfer Zap Channel
\end{verbatim}
\subsection{Authority}
\begin{verbatim}
<none>
\end{verbatim}
\subsection{Description}
\begin{verbatim}
(null)
\end{verbatim}


