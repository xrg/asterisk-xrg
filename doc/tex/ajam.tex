\section{Asynchronous Javascript Asterisk Manger (AJAM)}

AJAM is a new technology which allows web browsers or other HTTP enabled
applications and web pages to directly access the Asterisk Manger
Interface (AMI) via HTTP.  Setting up your server to process AJAM
involves a few steps:

\subsection{Setup the Asterisk HTTP server}

\begin{enumerate}
\item Uncomment the line "enabled=yes" in \path{/etc/asterisk/http.conf} to enable
   Asterisk's builtin micro HTTP server.

\item If you want Asterisk to actually deliver simple HTML pages, CSS,
   javascript, etc. you should uncomment "enablestatic=yes"

\item Adjust your "bindaddr" and "bindport" settings as appropriate for
   your desired accessibility

\item Adjust your "prefix" if appropriate, which must be the beginning of
   any URI on the server to match.  The default is "asterisk" and the
   rest of these instructions assume that value.
\end{enumerate}

\subsection{Allow Manager Access via HTTP}

\begin{enumerate}
\item Make sure you have both "enabled = yes" and "webenabled = yes" setup
   in \path{/etc/asterisk/manager.conf}

\item You may also use "httptimeout" to set a default timeout for HTTP
   connections.

\item Make sure you have a manager username/secret
\end{enumerate}

Once those configurations are complete you can reload or restart
Asterisk and you should be able to point your web browser to specific
URI's which will allow you to access various web functions.  A complete
list can be found by typing "http show status" at the Asterisk CLI.

examples:
\begin{astlisting}
\begin{verbatim}
http://localhost:8088/asterisk/manager?action=login&username=foo&secret=bar
\end{verbatim}
\end{astlisting}
This logs you into the manager interface's "HTML" view.  Once you're
logged in, Asterisk stores a cookie on your browser (valid for the
length of httptimeout) which is used to connect to the same session.
\begin{astlisting}
\begin{verbatim}
http://localhost:8088/asterisk/rawman?action=status
\end{verbatim}
\end{astlisting}
Assuming you've already logged into manager, this URI will give you a
"raw" manager output for the "status" command.
\begin{astlisting}
\begin{verbatim}
http://localhost:8088/asterisk/mxml?action=status
\end{verbatim}
\end{astlisting}
This will give you the same status view but represented as AJAX data,
theoretically compatible with RICO (\url{http://www.openrico.org}).
\begin{astlisting}
\begin{verbatim}
http://localhost:8088/asterisk/static/ajamdemo.html
\end{verbatim}
\end{astlisting}
If you have enabled static content support and have done a make install,
Asterisk will serve up a demo page which presents a live, but very
basic, "astman" like interface.  You can login with your username/secret
for manager and have a basic view of channels as well as transfer and
hangup calls.  It's only tested in Firefox, but could probably be made
to run in other browsers as well.

A sample library (astman.js) is included to help ease the creation of
manager HTML interfaces.

Note that for the demo, there is no need for *any* external web server.

\subsection{Integration with other web servers}

Asterisk's micro HTTP server is *not* designed to replace a general
purpose web server and it is intentionally created to provide only the
minimal interfaces required.  Even without the addition of an external
web server, one can use Asterisk's interfaces to implement screen pops
and similar tools pulling data from other web servers using iframes,
div's etc.  If you want to integrate CGI's, databases, PHP, etc.  you
will likely need to use a more traditional web server like Apache and
link in your Asterisk micro HTTP server with something like this:
\begin{astlisting}
\begin{verbatim}
ProxyPass /asterisk http://localhost:8088/asterisk
\end{verbatim}
\end{astlisting}

