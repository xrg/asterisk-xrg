\section{Applications}

\begin{itemize}
    \item SetAccount - Set account code for billing
    \item SetAMAFlags - Sets AMA flags
    \item NoCDR - Make sure no CDR is saved for a specific call
    \item ResetCDR - Reset CDR
    \item ForkCDR - Save current CDR and start a new CDR for this call
    \item Authenticate - Authenticates and sets the account code
    \item SetCDRUserField - Set CDR user field
    \item AppendCDRUserField - Append data to CDR User field
\end{itemize}

For more information, use the "core show application $<$application$>$" command.
You can set default account codes and AMA flags for devices in
channel configuration files, like sip.conf, iax.conf etc.

\section{CDR Fields}

\begin{itemize}
   \item accountcode:  What account number to use, (string, 20 characters)
   \item src:  Caller*ID number (string, 80 characters)
   \item dst:  Destination extension (string, 80 characters)
   \item dcontext:  Destination context (string, 80 characters)
   \item clid:  Caller*ID with text (80 characters)
   \item channel:  Channel used (80 characters)
   \item dstchannel:  Destination channel if appropriate (80 characters)
   \item lastapp:  Last application if appropriate (80 characters)
   \item lastdata:  Last application data (arguments) (80 characters)
   \item start:  Start of call (date/time)
   \item answer:  Answer of call (date/time)
   \item end:  End of call (date/time)
   \item duration:  Total time in system, in seconds (integer), from dial to hangup
   \item billsec:  Total time call is up, in seconds (integer), from answer to hangup
   \item disposition:  What happened to the call: ANSWERED, NO ANSWER, BUSY
   \item amaflags:  What flags to use: DOCUMENTATION, BILL, IGNORE etc,
            specified on a per channel basis like accountcode.
   \item user field:  A user-defined field, maximum 255 characters
\end{itemize}

In some cases, uniqueid is appended:

\begin{itemize}
   \item uniqueid:  Unique Channel Identifier (32 characters)
      This needs to be enabled in the source code at compile time
\end{itemize}

NOTE: If you use IAX2 channels for your calls, and allow 'full' transfers
(not media-only transfers), then when the calls is transferred the server
in the middle will no longer be involved in the signaling path, and thus
will not generate accurate CDRs for that call. If you can, use media-only
transfers with IAX2 to avoid this problem, or turn off transfers completely
(although this can result in a media latency increase since the media packets
have to traverse the middle server(s) in the call).

\section{Variables}

If the channel has a CDR, that CDR has its own set of variables which can be
accessed just like channel variables. The following builtin variables are
available.

\begin{verbatim}
${CDR(clid)}         Caller ID
${CDR(src)}          Source
${CDR(dst)}          Destination
${CDR(dcontext)}     Destination context
${CDR(channel)}      Channel name
${CDR(dstchannel)}   Destination channel
${CDR(lastapp)}      Last app executed
${CDR(lastdata)}     Last app's arguments
${CDR(start)}        Time the call started.
${CDR(answer)}       Time the call was answered.
${CDR(end)}          Time the call ended.
${CDR(duration)}     Duration of the call.
${CDR(billsec)}      Duration of the call once it was answered.
${CDR(disposition)}  ANSWERED, NO ANSWER, BUSY
${CDR(amaflags)}     DOCUMENTATION, BILL, IGNORE etc
${CDR(accountcode)}  The channel's account code.
${CDR(uniqueid)}     The channel's unique id.
${CDR(userfield)}    The channels uses specified field.
\end{verbatim}

In addition, you can set your own extra variables by using Set(CDR(name)=value).
These variables can be output into a text-format CDR by using the cdr\_custom
CDR driver; see the cdr\_custom.conf.sample file in the configs directory for
an example of how to do this.
