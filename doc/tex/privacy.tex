So, you want to avoid talking to pesky telemarketers/charity
seekers/poll takers/magazine renewers/etc?

\subsection{First of all}

 the FTC "Don't call" database, this alone will reduce your
telemarketing call volume considerably.  (see:
\url{https://www.donotcall.gov/default.aspx} ) But, this list won't protect
from the Charities, previous business relationships, etc.

\subsection{Next, Fight against autodialers!!}

Zapateller detects if callerid is present, and if not, plays the
da-da-da tones that immediately precede messages like, "I'm sorry,
the number you have called is no longer in service."

Most humans, even those with unlisted/callerid-blocked numbers, will
not immediately slam the handset down on the hook the moment they hear
the three tones. But autodialers seem pretty quick to do this.

I just counted 40 hangups in Zapateller over the last year in my
CDR's.  So, that is possibly 40 different telemarketers/charities that have
hopefully slashed my back-waters, out-of-the-way, humble home phone
number from their lists.

I highly advise Zapateller for those seeking the nirvana of "privacy".


\subsection{Next, Fight against the empty CALLERID!}

A considerable percentage of the calls you don't want, come from
sites that do not provide CallerID.

Null callerid's are a fact of life, and could be a friend with an
unlisted number, or some charity looking for a handout. The
PrivacyManager application can help here. It will ask the caller to
enter a 10-digit phone number. They get 3 tries(configurable), and this is
configurable, with control being passed to priority+101 if they won't
supply one.

PrivacyManager can't guarantee that the number they supply is any
good, tho, as there is no way to find out, short of hanging up and
calling them back. But some answers are obviously wrong. For instance,
it seems a common practice for telemarketers to use your own number
instead of giving you theirs. A simple test can detect this. More
advanced tests would be to look for -555- numbers, numbers that count
up or down, numbers of all the same digit, etc.

My logs show that 39 have hung up in the PrivacyManager script over
the last year.

(Note: Demanding all unlisted incoming callers to enter their CID may
not always be appropriate for all users. Another option might be to
use call screening. See below.)


\subsection{Next, use a WELCOME MENU !}

Experience has shown that simply presenting incoming callers with
a set of options, no matter how simple, will deter them from calling
you. In the vast majority of situations, a telemarketer will simply
hang up rather than make a choice and press a key.

This will also immediately foil all autodialers that simply belch a
message in your ear and hang up.

\subsubsection{Example usage of Zapateller and PrivacyManager}

\begin{astlisting}
\begin{verbatim}
[homeline]
exten => s,1,Answer
exten => s,2,SetVar,repeatcount=0
exten => s,3,Zapateller,nocallerid
exten => s,4,PrivacyManager
     ;; do this if they don't enter a number to Privacy Manager
exten => s,105,Background(tt-allbusy)
exten => s,106,Background(tt-somethingwrong)
exten => s,107,Background(tt-monkeysintro)
exten => s,108,Background(tt-monkeys)
exten => s,109,Background(tt-weasels)
exten => s,110,Hangup
exten => s,5,GotoIf($[ "${CALLERID(num)}"  = "7773334444" & "${CALLERID(name)}" : "Privacy Manager" ]?callerid-liar,s,1:s,7)
\end{verbatim}
\end{astlisting}

I suggest using Zapateller at the beginning of the context, before
anything else, on incoming calls.This can be followed by the
PrivacyManager App.

Make sure, if you do the PrivacyManager app, that you take care of the
error condition! or their non-compliance will be rewarded with access
to the system. In the above, if they can't enter a 10-digit number in
3 tries, they get the humorous "I'm sorry, but all household members
are currently helping other telemarketers...", "something is terribly
wrong", "monkeys have carried them away...", various loud monkey
screechings, "weasels have...", and a hangup. There are plenty of
other paths to my torture scripts, I wanted to have some fun.

In nearly all cases now, the telemarketers/charity-seekers that
usually get thru to my main intro, hang up. I guess they can see it's
pointless, or the average telemarketer/charity-seeker is instructed
not to enter options when encountering such systems. Don't know.


\subsection{Next: Torture Them!}

I have developed an elaborate script to torture Telemarketers, and
entertain friends. (See
\url{http://www.voip-info.org/wiki-Asterisk+Telemarketer+Torture} )

While mostly those that call in and traverse my teletorture scripts
are those we know, and are doing so out of curiosity, there have been
these others from Jan 1st,2004 thru June 1st, 2004:
(the numbers may or may not be correct.)

\begin{itemize}
    \item 603890zzzz -- hung up telemarket options.
    \item "Integrated Sale" -- called a couple times. hung up in telemarket options
    \item "UNITED STATES GOV" -- maybe a military recruiter, trying to lure one of my sons.
    \item 800349zzzz -- hung up in charity intro
    \item 800349zzzz -- hung up in charity choices, intro, about the only one who actually travelled to the bitter bottom of the scripts!
    \item 216377zzzz -- hung up the magazine section
    \item 626757zzzz = "LIR    " (pronounced "Liar"?) hung up in telemarket intro, then choices
    \item 757821zzzz -- hung up in new magazine subscription options.
\end{itemize}

That averages out to maybe 1 a month. That puts into question whether
the ratio of the amount of labor it took to make the scripts versus
the benefits of lower call volumes was worth it, but, well, I had fun,
so what the heck.

but, that's about it. Not a whole lot. But I haven't had to say "NO"
or "GO AWAY" to any of these folks for about a year now ...!

\subsection{Using Call Screening}

Another option is to use call screening in the Dial command. It has
two main privacy modes, one that remembers the CID of the caller, and
how the callee wants the call handled, and the other, which does not
have a "memory".

Turning on these modes in the dial command results in this sequence of
events, when someone calls you at an extension:

\begin{enumerate}
\item The caller calls the Asterisk system, and at some point, selects an
option or enters an extension number that would dial your extension.

\item Before ringing your extension, the caller is asked to supply an
introduction. The application asks them: "After the tone, say your
name". They are allowed 4 seconds of introduction.

\item After that, they are told "Hang on, we will attempt to connect you
to your party. Depending on your dial options, they will hear ringing
indications, or get music on hold. I suggest music on hold.

\item Your extension is then dialed. When (and if) you pick up, you are
told that a caller presenting themselves as $<$their recorded intro is
played$>$ is calling, and you have options, like being connected,
sending them to voicemail, torture, etc.

\item You make your selection, and the call is handled as you chose.
\end{enumerate}

There are some variations, and these will be explained in due course.


To use these options, set your Dial to something like:
\begin{astlisting}
\begin{verbatim}
exten => 3,3,Dial(DAHDI/5r3&DAHDI/6r3,35,tmPA(beep))
    or
exten => 3,3,Dial(DAHDI/5r3&DAHDI/6r3,35,tmP(something)A(beep))
    or
exten => 3,3,Dial(DAHDI/5r3&DAHDI/6r3,35,tmpA(beep))
\end{verbatim}
\end{astlisting}

The 't' allows the dialed party to transfer the call using '\#'. It's
optional.

The 'm' is for music on hold. I suggest it. Otherwise, the calling
party gets to hear all the ringing, and lack thereof. It is generally
better to use Music On Hold. Lots of folks hang up after the 3rd or
4th ring, and you might lose the call before you can enter an option!

The 'P' option alone will database everything using the extension as a
default 'tree'. To get multiple extensions sharing the same database, use
P(some-shared-key). Also, if the same person has multiple extensions,
use P(unique-id) on all their dial commands.

Use little 'p' for screening. Every incoming call will include a
prompt for the callee's choice.

the A(beep), will generate a 'beep' that the callee will hear if they
choose to talk to the caller. It's kind of a prompt to let the callee
know that he has to say 'hi'. It's not required, but I find it
helpful.

When there is no CallerID, P and p options will always record an intro
for the incoming caller. This intro will be stored temporarily in the
\path{/var/lib/asterisk/sounds/priv-callerintros} dir, under the name
NOCALLERID\_$<$extension$>$ $<$channelname$>$ and will be erased after the
callee decides what to do with the call.

Of course, NOCALLERID is not stored in the database. All those with no
CALLERID will be considered "Unknown".

\subsection{The 'N' and 'n' options}

Two other options exist, that act as modifiers to the privacy options
'P' and 'p'. They are 'N' and 'n'. You can enter them as dialing
options, but they only affect things if P or p are also in the
options.

'N' says, "Only screen the call if no CallerID is present". So, if a
callerID were supplied, it will come straight thru to your extension.

'n' says, "Don't save any introductions". Folks will be asked to
supply an introduction ("At the tone, say your name") every time they
call. Their introductions will be removed after the callee makes a
choice on how to handle the call. Whether the P option or the p option
is used, the incoming caller will have to supply their intro every
time they call.


\subsection{Recorded Introductions}

\subsubsection{Philosophical Side Note}
The 'P' option stores the CALLERID in the database, along with the
callee's choice of actions, as a convenience to the CALLEE, whereas
introductions are stored and re-used for the convenience of the CALLER.

\subsubsection{Introductions}
Unless instructed to not save introductions (see the 'n' option above),
the screening modes will save the recordings of the caller's names in
the directory \path{/var/lib/asterisk/sounds/priv-callerintros}, if they have
a CallerID.  Just the 10-digit callerid numbers are used as filenames,
with a ".gsm" at the end.

Having these recordings around can be very useful, however...

First of all, if a callerid is supplied, and a recorded intro for that
number is already present, the caller is spared the inconvenience of
having to supply their name, which shortens their call a bit.

Next of all, these intros can be used in voicemail, played over
loudspeakers, and perhaps other nifty things. For instance:

\begin{astlisting}
\begin{verbatim}
exten => s,6,Set(PATH=/var/lib/asterisk/sounds/priv-callerintros)
exten => s,7,System(/usr/bin/play ${PATH}/${CALLERID(num)}.gsm&,0)
\end{verbatim}
\end{astlisting}

When a call comes in at the house, the above priority gets executed,
and the callers intro is played over the phone systems speakers. This
gives us a hint who is calling.

(Note: the ,0 option at the end of the System command above, is a
local mod I made to the System command. It forces a 0 result code to
be returned, whether the play command successfully completed or
not. Therefore, I don't have to ensure that the file exists or
not. While I've turned this mod into the developers, it hasn't been
incorporated yet. You might want to write an AGI or shell script to
handle it a little more intelligently)

And one other thing. You can easily supply your callers with an option
to listen to, and re-record their introductions. Here's what I did in
the home system's extensions.conf. (assume that a
Goto(home-introduction,s,1) exists somewhere in your main menu as an
option):

\begin{astlisting}
\begin{verbatim}
[home-introduction]
exten => s,1,Background(intro-options) ;; Script:
     ;; To hear your Introduction, dial 1.
     ;;    to record a new introduction, dial 2.
     ;;    to return to the main menu, dial 3.
     ;;    to hear what this is all about, dial 4.
exten => 1,1,Playback,priv-callerintros/${CALLERID(num)}
exten => 1,2,Goto(s,1)
exten => 2,1,Goto(home-introduction-record,s,1)
exten => 3,1,Goto(homeline,s,7)
exten => 4,1,Playback(intro-intro)
     ;; Script:
     ;; This may seem a little strange, but it really is a neat
     ;; thing, both for you and for us. I've taped a short introduction
     ;; for many of the folks who normally call us. Using the Caller ID
     ;; from each incoming call, the system plays the introduction
     ;; for that phone number over a speaker, just as the call comes in.
     ;; This helps the folks
     ;; here in the house more quickly determine who is calling.
     ;; and gets the right ones to gravitate to the phone.
     ;; You can listen to, and record a new intro for your phone number
     ;; using this menu.
exten => 4,2,Goto(s,1)
exten => t,1,Goto(s,1)
exten => i,1,Background(invalid)
exten => i,2,Goto(s,1)
exten => o,1,Goto(s,1)

[home-introduction-record]
exten => s,1,Background(intro-record-choices)    ;; Script:
     ;; If you want some advice about recording your
     ;; introduction, dial 1.
     ;; otherwise, dial 2, and introduce yourself after
     ;; the beep.
exten => 1,1,Playback(intro-record)
     ;; Your introduction should be short and sweet and crisp.
     ;; Your introduction will be limited to 4 seconds.
     ;; This is NOT meant to be a voice mail message, so
     ;; please, don't say anything about why you are calling.
     ;; After we are done making the recording, your introduction
     ;; will be saved for playback.
     ;; If you are the only person that would call from this number,
     ;; please state your name.  Otherwise, state your business
     ;; or residence name instead. For instance, if you are
     ;; friend of the family, say, Olie McPherson, and both
     ;; you and your kids might call here a lot, you might
     ;; say: "This is the distinguished Olie McPherson Residence!"
     ;; If you are the only person calling, you might say this:
     ;; "This is the illustrious Kermit McFrog! Pick up the Phone, someone!!"
     ;; If you are calling from a business, you might pronounce a more sedate introduction,like,
     ;; "Fritz from McDonalds calling.", or perhaps the more original introduction:
     ;; "John, from the Park County Morgue. You stab 'em, we slab 'em!".
     ;; Just one caution: the kids will hear what you record every time
     ;; you call. So watch your language!
     ;; I will begin recording after the tone.
     ;; When you are done, hit the # key. Gather your thoughts and get
     ;; ready. Remember, the # key will end the recording, and play back
     ;; your intro. Good Luck, and Thank you!"
exten => 1,2,Goto(2,1)
exten => 2,1,Background(intro-start)
     ;;  OK, here we go! After the beep, please give your introduction.
exten => 2,2,Background(beep)
exten => 2,3,Record(priv-callerintros/${CALLERID(num)}:gsm,4)
exten => 2,4,Background(priv-callerintros/${CALLERID(num)})
exten => 2,5,Goto(home-introduction,s,1)
exten => t,1,Goto(s,1)
exten => i,1,Background(invalid)
exten => i,2,Goto(s,1)
exten => o,1,Goto(s,1)
\end{verbatim}
\end{astlisting}

In the above, you'd most likely reword the messages to your liking,
and maybe do more advanced things with the 'error' conditions (i,o,t priorities),
but I hope it conveys the idea.


