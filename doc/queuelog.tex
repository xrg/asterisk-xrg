In order to properly manage ACD queues, it is important to be able to
keep track of details of call setups and teardowns in much greater detail
than traditional call detail records provide.  In order to support this,
extensive and detailed tracing of every queued call is stored in the
queue log, located (by default) in /var/log/asterisk/queue\_log.

These are the events (and associated information) in the queue log:

ABANDON(position|origposition|waittime)
The caller abandoned their position in the queue.  The position is the
caller's position in the queue when they hungup, the origposition is
the original position the caller was when they first entered the
queue, and the waittime is how long the call had been waiting in the 
queue at the time of disconnect.

AGENTDUMP
The agent dumped the caller while listening to the queue announcement.

AGENTLOGIN(channel)
The agent logged in.  The channel is recorded.

AGENTCALLBACKLOGIN(exten@context)
The callback agent logged in.  The login extension and context is recorded.

AGENTLOGOFF(channel|logintime)
The agent logged off.  The channel is recorded, along with the total time
the agent was logged in.

AGENTCALLBACKLOGOFF(exten@context|logintime|reason)
The callback agent logged off.  The last login extension and context is
recorded, along with the total time the agent was logged in, and the
reason for the logoff if it was not a normal logoff 
(e.g., Autologoff, Chanunavail)

COMPLETEAGENT(holdtime|calltime|origposition)
The caller was connected to an agent, and the call was terminated normally
by the *agent*.  The caller's hold time and the length of the call are both
recorded.  The caller's original position in the queue is recorded in
origposition.

COMPLETECALLER(holdtime|calltime|origposition)
The caller was connected to an agent, and the call was terminated normally
by the *caller*.  The caller's hold time and the length of the call are both
recorded.  The caller's original position in the queue is recorded in
origposition.

CONFIGRELOAD
The configuration has been reloaded (e.g. with asterisk -rx reload)

CONNECT(holdtime|bridgedchanneluniqueid)
The caller was connected to an agent.  Hold time represents the amount
of time the caller was on hold. The bridged channel unique ID contains
the unique ID of the queue member channel that is taking the call. This
is useful when trying to link recording filenames to a particular
call in the queue.

ENTERQUEUE(url|callerid)
A call has entered the queue.  URL (if specified) and Caller*ID are placed
in the log.

EXITEMPTY(position|origposition|waittime)
The caller was exited from the queue forcefully because the queue had no
reachable members and it's configured to do that to callers when there
are no reachable members. The position is the caller's position in the
queue when they hungup, the origposition is the original position the 
caller was when they first entered the queue, and the waittime is how 
long the call had been waiting in the queue at the time of disconnect.

EXITWITHKEY(key|position|origposition|waittime)
The caller elected to use a menu key to exit the queue.  The key and
the caller's position in the queue are recorded.  The caller's entry
position and amoutn of time waited is also recorded.

EXITWITHTIMEOUT(position|origposition|waittime)
The caller was on hold too long and the timeout expired.  The position in the
queue when the timeout occurred, the entry position, and the amount of time
waited are logged.

QUEUESTART
The queueing system has been started for the first time this session.

RINGNOANSWER(ringtime)
After trying for ringtime ms to connect to the available queue member,
the attempt ended without the member picking up the call. Bad queue
member!

SYSCOMPAT
A call was answered by an agent, but the call was dropped because the 
channels were not compatible.

TRANSFER(extension|context|holdtime|calltime)
Caller was transferred to a different extension.  Context and extension
are recorded. The caller's hold time and the length of the call are both
recorded. PLEASE remember that transfers performed by SIP UA's by way
of a reinvite may not always be caught by Asterisk and trigger off this
event. The only way to be 100\% sure that you will get this event when
a transfer is performed by a queue member is to use the built-in transfer
functionality of Asterisk.

